%%%%%%%%%%%%%%%%%%%%%%%%%%%%%%%%%%%%%%%%%%%%%%%%%%%%%%%%%%%%%%%%%%%%%%%%%%%%%%%
\documentclass[11pt,a4paper]{ctexart}

\usepackage{color,xcolor,dsfont}
\usepackage[all]{xy}
\usepackage{epsfig}
\usepackage{amssymb}
\usepackage{amsmath}
\usepackage{graphicx}
\usepackage{enumerate}
\usepackage{extarrows}
\usepackage{ntheorem}
\usepackage{mathrsfs}

\topmargin=0mm \evensidemargin=0mm \oddsidemargin=0mm \headsep=0mm
\textwidth=16cm \textheight=24cm
\parindent=2em
\pagestyle{plain}


%%%%%%%%%%%%%%%%%%%%%%%%%%%%%%%%%%%%%%%%%%%%%%%%%%%%%%%%%%%%%%%%%%%%%%%%%%%%%%%%%

\begin{document}

\begin{center}
\Large\textbf{Liu's Solutions}
\end{center}
\noindent 1.\\
(a)Yes. Firstly we can check that $P(X)$ has no roots in $\mathbb{F}_2$, so the only way it could not be irreducible is by being a product of two irreducible polynomials of degree 2. But the only irreducible polynomials of degree 2 in $\mathbb{F}_2[X]$ is $X^2+X+1$, and $P$ is not its square.\\
(b) No. We can check that $P(0)=P(2)=P(3)=1,P(1)=3$, so $P(X)$ has no roots in $\mathbb{F}_4$.\\
(c) No. If it is irreducible, the it will have a root in $\mathbb{F}_{16}$, which is impossible by $(d)$.\\
(d) Yes. Notice that $\mathbb{F}_8$ is a degree 3 extension of $\mathbb{F}_2$ and 3 is prime to 4, so by the lecture the polynomial remains irreducible in $\mathbb{F}_8$.\\
(e) Yes. $\mathbb{F}_{16}$ is its splitting field according to the lecture.\\
(f) No. Notice that $X^4+X^3$ will always be even if $X\in\mathbb{Z}$, $P(X)$ are all odd number in $\mathbb{F}_{32}$ on integer points between 0 and 31, so $P(X)$ has no roots in $\mathbb{F}_{32}$.\\
(g) No. Notice that $X^4+X^3$ will always be even if $X\in\mathbb{Z}$, $P(X)$ are all odd number in $\mathbb{F}_{64}$ on integer points between 0 and 63, so $P(X)$ has no roots in $\mathbb{F}_{64}$.\\
(h) No. Since $\mathbb{F}_{64}$ contains $\mathbb{F}_4$ and our polynomial is the product of two quadratic factors over $\mathbb{F}_4$, it is not irreducible in $\mathbb{F}_{64}$.\vspace{10mm}\\

\noindent 2.\\
(a) Now denote $f(X)$ that polynomial. Consider
$f(X+1)=\frac{(X+1)^p-1}{X}=X^{p-1}+p X^{p-2}+\cdots+p$
Then use Eisenstein's criterion by $p$ to reach our conclusion.\\
(b) By (a), the polynomial $X^6+X^5+c\dots+X+1$ is irreducible and has $\xi$ as a root, so it must be the minimal polynomial of $\xi$. Therefore $[L:\mathbb{Q}]=6$.\\
(c) We could check the polynomial $(X-\xi)(X-\frac{1}{\xi})=x^2-2\cos{\frac{2\pi}{7}}x+1$ is irreducible in $M$ and has $\xi$ as a root, so it must be the minimal polynomial of $\xi$ over $M$. Therefore $[L:M]=2$ and hence $[M,\mathbb{Q}]=6$.\\
(d) An automorphism $f$ of $L$ must satisfy $f(\xi)=\xi^k(k=1,2,3,4,5,6)$. Moreover, $f(\cos{\frac{2\pi}{7}})=\frac{\xi^{2k}+1}{\xi^k}=\cos{\frac{2k\pi}{7}}$, where $k=1,2,3$.\vspace{10mm}\\

\noindent 3.\\
(a) Notice that $\mathbb{Q}(2^{1/3})\cong\mathbb{Q}[t]/(t^3-2)$, we have
$$\mathbb{Q}(\sqrt{2})\otimes_{\mathbb{Q}}\mathbb{Q}(2^{1/3})\cong\mathbb{Q}(\sqrt{2})[t]/(t^3-2)$$
Since $t^3-2$ is irreducible over $\mathbb{Q}[\sqrt{2}]$, $\mathbb{Q}(\sqrt{2})[t]/(t^3-2)$ is a field extension of $\mathbb{Q}[\sqrt{2}]$.

\newpage
\noindent(b) Notice that $\mathbb{Q}(2^{1/4})\cong\mathbb{Q}[t]/(t^4-2)$, we have
$$\mathbb{Q}(\sqrt{2})\otimes_{\mathbb{Q}}\mathbb{Q}(2^{1/4})\cong\mathbb{Q}(\sqrt{2})[t]/(t^4-2)$$
Since $t^4-2=(t^2-\sqrt{2})(t^2+\sqrt{2})$ over $\mathbb{Q}[\sqrt{2}]$ and $t^2-\sqrt{2},t^2+\sqrt{2}$ are irreducible over $\mathbb{Q}[\sqrt{2}]$, we have
$$\mathbb{Q}[t]/(t^4-2)\cong\mathbb{Q}[t]/(t^2+\sqrt{2})\times\mathbb{Q}[t]/(t^2-\sqrt{2})$$
(c) Notice that $\mathbb{F}_2(\sqrt{T})\cong\mathbb{F}_2(T)[X]/(X^2-T)$, we have
$$\mathbb{F}_2(\sqrt{T})\otimes_{\mathbb{F}_2(T)}\mathbb{F}_2(\sqrt{T})\cong\mathbb{F}_2(T)[X]/(X^2-T)$$
Since $X^2-T=(X-\sqrt{T})^2$ over $\mathbb{F}_2(\sqrt{T})$, our algebra contains nilpotent and in this case it can never be a field or product of fields.\\
(d) Similarly, we have
$$\mathbb{F}_4(T^{1/3})\otimes_{\mathbb{F}_4(T)}\mathbb{F}_4(T^{1/3})\cong\mathbb{F}_4(T^{1/3})[X]/(X^3-T)$$
The polynomial $X^3-T$ is split over $\mathbb{F}_4(T^{1/3})$, with roots $T^{1/3},T^{1/3}\cdot\xi,T^{1/3}\cdot\xi^{-1}$, where $\xi$ is a root of $X^2+X+1$ in $\mathbb{F}_4$. Then we have(by chinese remainder theorem):
$$\mathbb{F}_4(T^{1/3})\otimes_{\mathbb{F}_4(T)}\mathbb{F}_4(T^{1/3})\cong\mathbb{F}_4(T^{1/3})\times\mathbb{F}_4(T^{1/3})\times\mathbb{F}_4(T^{1/3})$$

\end{document}
