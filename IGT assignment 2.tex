%%%%%%%%%%%%%%%%%%%%%%%%%%%%%%%%%%%%%%%%%%%%%%%%%%%%%%%%%%%%%%%%%%%%%%%%%%%%%%%
\documentclass[11pt,a4paper]{ctexart}

\usepackage{color,xcolor,dsfont}
\usepackage[all]{xy}
\usepackage{epsfig}
\usepackage{amssymb}
\usepackage{amsmath}
\usepackage{graphicx}
\usepackage{enumerate}
\usepackage{extarrows}
\usepackage{ntheorem}
\usepackage{mathrsfs}

\topmargin=0mm \evensidemargin=0mm \oddsidemargin=0mm \headsep=0mm
\textwidth=16cm \textheight=24cm
\parindent=2em
\pagestyle{plain}

\makeatother
\renewcommand\contentsname{Contents}

\newtheorem{thm}{Theorem}[subsection]
\newtheorem{lema}[thm]{Lemma}
\newtheorem{ppst}[thm]{Proposition}
\newtheorem{coro}[thm]{Corollary}
\newenvironment{proof}{{\noindent\it Proof}\quad}{\hfill $\square$\par}


%%%%%%%%%%%%%%%%%%%%%%%%%%%%%%%%%%%%%%%%%%%%%%%%%%%%%%%%%%%%%%%%%%%%%%%%%%%%%%%





\begin{document}

\begin{center}
\Large\textbf{Liu's Solutions}
\end{center}
\noindent 1.\\
(a) Let $F=\mathbb{Q}(\xi)$, where $\xi=\textup{e}^{2\pi i/9}$. Notice that
$$x^9-1=(x-1)(x^2+x+1)(x^6+x^3+1)$$
and we can check that the root of $x-1$ is $1$, the roots of $x^2+x+1$ are $\xi^3,\xi^6$, so roots of $x^6+x^3+1$ are $\xi,\xi^8,\xi^2,\xi^7,\xi^4,\xi^5$.\\
In this case, $x^6+x^3+1$ has no roots in $\mathbb{Q}$, so if it is reducible, it would have rational factor of degree $2$ or of degree $3$. But if it had a rational factor of degree 2, the factor would have the form
$$(x-\xi^k)(x-\xi^{-k})=x^2-2\cos{2k\pi/9}x+1$$
which is not rational. And if it had a rational factor of degree 3, it must have a real root, which contradicts to what we have known. Therefore, the polynomial $x^6+x^3+1$ is reducible and the extension $[F=\mathbb{Q}(\xi):\mathbb{Q}]=6$. Moreover, $F=\mathbb{Q}[\xi]$ is the splitting field of $x^6+x^3+1$ and $F/\mathbb{Q}$ is a Galois extension.\\
It is not hard for us to get that Galois Group Gal$(F/\mathbb{Q})$ is isomorphic to $\mathbb{Z}_6$.\\
(b) Let $\alpha=\frac{\xi^2+1}{2\xi}=\cos{\frac{2\pi}{9}}$. Then the quadratic polynomial
$$(x-\xi)(x-\xi^{-1})=x^2-2\alpha x+1$$
is defined over $\mathbb{Q}(\alpha)$. The roots of it are non-real so $x^1-2\alpha x+1$ is irreducible. Thus it is the minimal polynomial of $\xi$ over $\mathbb{Q}(\alpha)$, and the extension $F/\mathbb{Q}(\alpha)$ has degree 2.\\
(c) Let $\beta$ be the real root of $x^9-5$, which is obviously irreducible. So $[\mathbb{Q}(\beta):\mathbb{Q}]=9$. Suppose field $K$ satisfies $\mathbb{Q}\subset K\subset\mathbb{\beta}$, it must be of degree 3 because $[\mathbb{Q}(\beta):K][K:\mathbb{Q}]=3$. In this case, $k/\mathbb{Q}$ has degree 3, and hence we have $K=\mathbb{Q}(\beta^3)$.\\
(d) We know that $[F\cap L:\mathbb{Q}]$ must be 3 or 1, where $L=\mathbb{Q}(\beta)$ in (c). If $[F\cap L:\mathbb{Q}]=3$, then by we know $F\cap L=K$. But by (b), we would have $\mathbb{Q}(\alpha)=\mathbb(\beta^3)$, which leads to contradiction. Because $\mathbb{Q}(\alpha)$ contains all the roots of the minimal polynomial of $\beta^3$ while $\mathbb{Q}(\beta^3)$ dose not.\\
In this case, $F\cap L=\mathbb{Q}$. Now, we can clearly see $M=L(\xi)=\mathbb{Q}(\beta,\xi)$. Since $x^6+x^3+1$ is irreducible over $L$, it is the minimal polynomial of $\xi$. Therefore, $[M:L]=6$ and then $[M:\mathbb{Q}]=[M:L][L:\mathbb{Q}]=54$.\\
(e) By (d), $[M:F]=9$ and $x^9-5$ is irreducible over $F$. Then $M/\mathbb{Q},F/\mathbb{Q}$ and $M/F$ are Galois and hence $H=\textup{Gal}(M/F)$ is a normal subgroup of Gal$(M/\mathbb{Q})$ of order 9, which is cyclic referring to the lecture.\\
Similarly, the extension $M/L$ is Galois of degree 6 with $S=\textup{Gal}(M/L)\subset\textup{Gal}(M/\mathbb{Q})$ of order 6.
Moreover $G$ is clearly non-commutative or every subgroup of $G$ would be normal, which is not true.

\newpage
\noindent(f) $E$ must be a fixed field of an index 2 subgroup $N$ of $G$ = Gal$(M/\mathbb{Q})$. Since subgroups of index 2 are always normal, there must be a surjective homomorphism $G\rightarrow \mathbb{Z}_2$ with kernel $N$.\\
In this case, $\varphi,\psi\in N$ and the subgroup of $G$ generated by $\varphi$ and $\psi^2$ has index 2 by (e). Therefore, $N=(\varphi,\psi^2)$.\\
So, $E$ must be of degree 2 over $\mathbb{Q}$ and fixed by $\varphi$ and $\psi^2$. Since $\xi^3=\textup{e}^{2\pi i/3}$ fixed by $\varphi$ and $\psi^2$ has minimal polynomial $x^2+x=1$, $E=\mathbb{\xi^3}$.
(g) The $E$ that $\mathbb{Q}\subset E\subset M$ of degree 3 over $\mathbb{Q}$ must be the fixed field of an index normal subgroup $N$ of $G$. Thus there must be a surjective homomorphism $G\rightarrow\mathbb{Z}_3$ with kernel $N$.\\
Under this homomorphism, $\psi^3$ must map to 0 and $\psi$ can not(because the only degree 3 subfield of $L$ is $\mathbb{Q}(\beta^3)$ which is not a Galois group). Therefore we could obtain that $N=<\varphi,\psi^3>$.\\
Since $\varphi\in N$, the fixed field $E$ of $N$ is contained in the fixed field of $\varphi$, which is $F$. $E$ is fixed by $\psi^3$, similarly. So we have $E\subset F\cap\mathbb{R}=\mathbb{Q}(\alpha)$, which implies that $E=\mathbb{Q}(\alpha)$.
\vspace{10mm}\\

\noindent 2.\\
(a) We can easily obtain that $X^{p-1}-a$ is split in $K$, which roots together with 0 are roots of $X^p-aX$. In characteristic $p$, the map $X\mapsto X^p-aX$ is a homomorphism of $\mathbb{F}_p$-vector spaces, so its kernel is an $\mathbb{F}_p$-vector space of dimension one, which is a cyclic group of order $p$.\\
(b) By Kummer theory, this is cyclic of order dividing $p-1$.\\
(c) $gx$ is also a root of $P$ and the difference of two roots of $P$ is 0 or a root of $X^{p-1}-a$. We have
$$(gx_1-x_1)-(gx_2-x_2)=g(x_1-x_2)-(x_1-x_2)$$
where $g\in H$ and $x_1,x_2$ two roots of $P$. Notice that $x_1-x_2\in L$ and $g$ is identity on $L$, so the above is zero.\\
(d) For $g\in H$, define $f(g)=gx-x$ for $x$ a root of $P$, which is an injective group homomorphism $f:H\rightarrow\mathbb{Z}_p$.
Since $p$ is a prime, the image of $f$ is either zero and $\mathbb{Z}_p$.\\
(e) The stem field of $P$ over $L$ is also its splitting field. If $H=\mathbb{Z}_p$ then the degree of the stem field of $P$ over $L$ is equal to the degree $p$ of $P$, meaning that $P$ is irreducible over $L$. If $P$ was not irreducible over $k$, then the degree of $K$ would have all its prime divisors less than $p$. However, it should be divisible by $p$ since $P$ is irreducible over $L$. Finally, if $P$ is irreducible over $k$, we obtain that $H$ has $p$ elements by the same divisibility argument.\\
(f) $P(X)=X^p-TX-T$ and $X^{p-1}-T$ are irreducible over $k$, which implies that $[L:k]=p-1$ and $[K:L]=p$, so the total degree is $p(p-1)$ and the order of Galois group is the same.

\end{document}
